Given the description of the dataset and the total number of samples (48842), the dataset is potentially the UCI Adult Data set~\cite{lichman2013}. This would underline the pattern of missing data. A possible method would be to use the labeled samples of the UCI dataset and match them with the given testset, which should give an accuracy of 1.0. However, this method is not as straight-forward as it may seem. The UCI dataset is split in a different manner into training and testset (amount of training samples: 32561, test samples: 16281). Additionally, the categorical variables in the UCI dataset are encoded as strings, while in the given dataset, they are encoded as integers. Therefore, one would have to find a mapping from strings to integers. The large amount of missing data might impede this endeavor. Since using the UCI Adult dataset might defeat the goal of this assignment, I did not take any steps in this direction. A comparison on \url{http://www.cs.toronto.edu/~delve/data/adult/adultDetail.html} shows that the best performing classifier is a Forward Sequential Selection (FSS) naive Bayes model with an accuracy of $85.95\,\%$. The classifiers were trained after removing unknown values (7\,\% of values had missing values; training: 30162, test: 15060). This accuracy can be used as an indicator of a good performance on the given dataset. However, there are two differences to the given dataset: (i) the original UCI dataset had a lower amount of missing data, and (ii) the UCI dataset had a higher number of training samples.