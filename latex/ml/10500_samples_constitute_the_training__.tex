10500 samples constitute the training set of which 2500 (23\,\%) were labeled as wealthy and 8000 (77\,\%) as non-wealhty. If we assume that the training and test set were randomly sampled from the entire dataset, therefore, the predictions on the test set should reflect this ratio. Additionally always predicting non-wealthy should give a 0-1 loss of 77\,\% under this assumption, which can be used as a baseline for the classifier performance. The assumption can be tested by probing the testset with a ``0-only" submission. This ratio might be useful for setting the class-weights of a classifier or for determining the decision threshold in a decision function: if a classifier is able to output probabilities for the class labels, one would predict ``label 1'', if the probability for ``wealthy'' is greater than the threshold $\theta = 0.5$. However, $\theta$ could be modified, to increase or decrease the amount of ``label 1'' predictions.