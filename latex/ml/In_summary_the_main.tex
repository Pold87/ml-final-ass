In summary, the main challenge lay in handling the missing data values.  Afterward, standard classification techniques could be used. The use of different classifiers had no substantial influence on the achieved accuracy: using a support vector machine, random forest regression, and AdaBoost
classifier resulted in very similar results, as can be seen in Table~\ref{tab:localpublic}. The use of semi-supervised methods did not have a substantial effect on the classification performance; this is in line with findings other studies, showing that semi-supervised methods do not necessarily lead to improved performances and can even lead to a lower accuracy~\cite{zhu2005semi}. It might be that semi-supervised methods would have had a greater effect, if the number of training samples would have been substantially smaller. 

Since the ground truth labels of the test set were not revealed, error statistics could only be done based on the 0-1 loss of the public leaderboard and on the local cross-validation. The restriction of one submission per day increased the difficulty of the validation. My public score of $85.494\,\%$---which is only $0.00364$ below the best performing score in the competition ($85.854\,\%$) and $0.00456$ below the best performing method of the Delve project~\footnote{\url{http://www.cs.toronto.edu/~delve/data/adult/adultDetail.html}}---is an indicator that little improvement will be possible beyond my system. Compared to the UCI dataset, the given dataset had a smaller amount of labeled data and more missing values, which made the Kaggle classification task more challenging. Using standard state-of-the-art classifiers, the maximum possible performance of the given dataset will be possibly capped clearly below 100\,\% due to mistakes during data acquisition, such as deliberate misinformation or transcription errors. Moreover, the used variables will only have limited explanatory power:  additional variables, such as political view, morale, or number of children might be needed to increase the amount of explainable variation.     

Due to the concealment of categorical variables, I did not consider manual feature engineering. Otherwise, it might have been possible to group certain levels of the categorical (dummy) variables.

Many steps in the classification pipeline were rather time-consuming and took several hours to complete. In the future, more processing power could help to increase the classification performance. For example, using a larger amount of imputed datasets could capture the uncertainty in the imputed values to a greater degree. The number of base classifiers---the decision stumps---in the AdaBoost classifier could be further increased. Moreover, the CPLE framework could be tested with classifiers that showed good performance in the cross-validation without transductive learning, such as random forests or support vector machines.    