#\documentclass{article}

\usepackage{booktabs}

\usepackage{tikz}
\usetikzlibrary{shapes,arrows,positioning}

\title{Machine Learning -- Final Report}
\author{Volker Strobel}
\date{\today}


% THE REPORT

% In addition to participating in the competition, we expect you to
% provide us with a report on your work that at least contains the
% following :

% - A clear description of the complete and, possibly, best-performing
% method you implemented;

% - Clear arguments for the different choices you made in compiling
% your classifier;

% - Sufficient experimental results, learning curves, error bars, or
% whatever else you need to convince the reader that little
% improvement will be possible beyond the system that your method is
% currently performing. Your final score is of course also there to
% see whom of your colleagues you have beaten and, for us, to use in
% the final decision on your grade.

% Please refer to the submission.csv file for an example of the suesent;

% - Any references that have been used.


\begin{document}
\maketitle

\begin{abstract}
  This reports presents the techniques and results of a classification
  problem involving missing data, a mixture of categorical and 
\end{abstract}

\section{Introduction \& Statement of Problem}

Predicting the challenge is to predict, whether a person earns over
EUR 40k a year.

In the following section, Section~2, we analyze and visualize the
structure of the data. In Section~33

\section{Analysis}

In order to motivate later classifier and technique choices, we start
with an in-depth analysis of the given data sets.

The used loss is the 1/0 loss.


\begin{table}[h]
  \centering
  \begin{tabular}{cc}
    \toprule
  Categorical & Continuous\\
    \midrule
    work class & age\\
    education & number of years of education\\
marital status & income from investment sources\\
occupation & losses from investment sources\\
relationship & working hours per week \\
race & \\
sex & \\
native country & \\
      \bottomrule
  \end{tabular}
  \caption{Overview of the used features}
  \label{tab:features}
\end{table}

\section{Methods}

Therefore, this competition involves three main challenges:
\begin{itemize}
\item Number of missing data points
\item Mixture of categorical and continuous variables
\item Classification of the output variable
\end{itemize}

We will address each of them in turn.

\section{Missing data values}

Missing data values are a common problem in machine learning
problems. The failure of sensors, or the conscious loss due to
anonymity impede the machine learning accuracy. While the
\emph{imputation} of these missing data is still a open problem,
several method have been put forth. For the competition, a
maximum-likelihood (expectation-maximization) method has been used.

\begin{figure}[h]
  \centering
% Define block styles
\tikzstyle{decision} = [diamond, draw, fill=blue!20, 
    text width=4.5em, text badly centered, node distance=3cm, inner sep=0pt]
\tikzstyle{block} = [rectangle, draw, 
    text width=5em, text centered, rounded corners, minimum
    height=4em, node distance=3cm]
\tikzstyle{line} = [draw, -latex']
\tikzstyle{cloud} = [draw, ellipse,fill=red!20, node distance=3cm,
    minimum height=2em]
  \begin{tikzpicture}[node distance = 2cm, auto]
    % Place nodes
    \node [block] (train) {NaN Training set};
    \node [block, below of=train] (test) {NaN Test set};
    \node [block, right of=train] (stacked) {NaN Combined set};
    \node [block, right of=stacked] (full) {Full Combined set};
    % Draw edges
    \path [line] (train) -- (stacked);
    \path [line] (test) -- (stacked);
\end{tikzpicture}
  \caption{The pipeline}
  \label{fig:pipeline}
\end{figure}

\end{document}